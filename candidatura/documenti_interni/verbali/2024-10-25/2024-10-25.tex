\documentclass[10pt]{article}
\usepackage{graphicx}
\usepackage[a4paper, margin=1in]{geometry}
\usepackage{helvet}
\usepackage{hyperref}
\usepackage[italian]{babel}
\usepackage{tabularx}
\hypersetup{
    colorlinks=true,
    linktoc=all,
    linkcolor=black
}

\title{Verbale interno del meeting in data 25/10/2024}
\author{Tech Minds \\ \href{mailto:techminds.unipd@gmail.com}{techminds.unipd@gmail.com}}
\date{}

\begin{document}

\maketitle

\begin{figure}[h]
    \centering
    \includegraphics[width=0.3\linewidth]{../../../../assets/logo.png}
\end{figure}

\tableofcontents{\newpage}

\section{Informazioni introduttive}
\subsection{Durata e luogo}
\begin{itemize}
  \item Inizio: 10:20
  \item Fine: 12:00
  \item Luogo: Aula 2AB40 Torre Archimede
\end{itemize}
\subsection{Partecipanti}
\begin{itemize}
    \item Lazzarin Tommaso
    \item Corradin Samuele
    \item Salviato Leonardo
    \item Tutino Giuseppe
    \item Bressan Alessandro
    \item Squarzoni Matteo
\end{itemize}
\section{Ordine del giorno}
L'ordine del giorno è il seguente:
\begin{enumerate}
    \item Discussione sui capitolati e revisione tra pari del documento delle valutazioni;
    \item Creazione delle diapositive necessarie per il diario di bordo;
    \item Stesura del documento di valutazione dei capitolati in linguaggio LaTeX;
    \item Varie ed eventuali.
\end{enumerate}
\section{Riassunto}
Il gruppo ha deciso di riunirsi in presenza per discutere dei vari capitolati e per iniziare a scegliere il progetto per cui candidarsi.\\
Per prima cosa si è scelto di dividersi in coppie per fare una revisione tra pari dell'analisi del capitolato che ognuno ha deciso di studiare.\\
Successivamente è stata effettuata la trascrizione in linguaggio LaTeX del documento di revisione dei capitolati, che verrà inserita nella repository.\\
È stato deciso di utilizzare Canva per creare la presentazione del diario di bordo perché è uno strumento completo che permette la collaborazione tra più utenti.
Il contenuto e la composizione della presentazione verranno scelti in maniera asincrona durante il resto della giornata. 


\section{Changelog}
\begin{tabularx}{0.8\textwidth} {
  | >{\centering\arraybackslash}X
  | >{\centering\arraybackslash}X
  | >{\centering\arraybackslash}X
  | >{\centering\arraybackslash}X | }
 \hline
 \textbf{Versione} & \textbf{Data} & \textbf{Descrizione} & \textbf{Autore} \\
 \hline
 1.0 & 25/10/2024 & Prima versione & Lazzarin Tommaso\\
\hline
\end{tabularx}

\end{document}
