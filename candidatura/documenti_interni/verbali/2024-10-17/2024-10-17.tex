\documentclass[10pt]{article}
\usepackage{graphicx}
\usepackage[a4paper, margin=1in]{geometry}
\usepackage{helvet}
\usepackage{hyperref}
\usepackage[italian]{babel}
\usepackage{tabularx}
\hypersetup{
    colorlinks=true,
    linktoc=all,
    linkcolor=black
}

\title{Verbale interno del meeting in data 17/10/2024}
\author{Tech Minds \\ \href{mailto:techminds.unipd@gmail.com}{techminds.unipd@gmail.com}}
\date{}

\begin{document}

\maketitle

\begin{figure}[h]
    \centering
    \includegraphics[width=0.3\linewidth]{../../../../assets/logo.png}
\end{figure}

\tableofcontents{\newpage}

\section{Durata e partecipanti}
\subsection{Durata}
\begin{itemize}
  \item Inizio: 14:45
  \item Fine: 16:30
\end{itemize}
\subsection{Partecipanti}
\begin{itemize}
    \item Lazzarin Tommaso
    \item Salviato Leonardo
    \item Squarzoni Matteo
    \item Vallotto Caterina
    \item Tutino Giuseppe
    \item Bressan Alessandro
    \item Corradin Samuele
  \end{itemize}

\section{Riassunto}
L'incontro è iniziato nell'orario previsto sulla piattaforma Discord. 
Il primo argomento di discussione è stato il confronto e il perfezionamento delle domande di chiarimento da presentare alle aziende.
\\Successivamente sono state inviate tre mail per chiedere i colloqui alle aziende dei capitolati C1, C3, C5 e altre due mail per chiedere chiarimenti rispettivamente a C2 e C4.
\\In seguito è stato creato un documento condiviso in cui valutare vari aspetti di tutti i capitolati tra cui punti di forza e di debolezza e altri tipi di considerazioni utili alla scelta del capitolato migliore per il gruppo.
\\Durante la discussione il referente del capitolato C1 ha proposto di fissare un incontro informativo Lunedì 21 Ottobre alle 14:30 sulla piattaforma Meet.
\\Il prossimo incontro con tutti i componenti del gruppo si terrà dopo l'ultimo incontro con le aziende contattate.
\section{Changelog}
\begin{tabularx}{0.8\textwidth} {
  | >{\centering\arraybackslash}X
  | >{\centering\arraybackslash}X
  | >{\centering\arraybackslash}X
  | >{\centering\arraybackslash}X | }
 \hline
 \textbf{Versione} & \textbf{Data} & \textbf{Descrizione} & \textbf{Autore} \\
 \hline
 1.0 & 17/10/2024 & Prima versione & Lazzarin Tommaso\\
\hline
\end{tabularx}

\end{document}
