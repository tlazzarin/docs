\documentclass[10pt]{article}
\usepackage{graphicx}
\usepackage[a4paper, margin=1in]{geometry}
\usepackage{helvet}
\usepackage{hyperref}
\usepackage[italian]{babel}
\usepackage{tabularx}
\hypersetup{
    colorlinks=true,
    linktoc=all,
    linkcolor=black
}

\title{\textbf{Verbale interno del 15/10/2024}}
\author{\href{mailto:techminds.unipd@gmail.com}{techminds.unipd@gmail.com}}
\date{}

\begin{document}

\begin{figure}
    \centering
    \includegraphics[width=0.8\linewidth]{../../../../assets/logo_upscaled.png}
\end{figure}
\maketitle
\begin{center}

  \textbf{Sommario}\\
  \vspace{3mm}
  Questo verbale documenta la prima riunione interna avvenuta il 15/10/2024. 
\end{center}
\newpage
\tableofcontents{\newpage}

\section{Informazioni introduttive}
\subsection{Durata e luogo}
\begin{itemize}
  \item Inizio: 15:10
  \item Fine: 16:15
  \item Luogo: chiamata Discord
\end{itemize}
\subsection{Partecipanti}
\begin{itemize}
  \item Lazzarin Tommaso
  \item Salviato Leonardo
  \item Squarzoni Matteo
  \item Vallotto Caterina
  \item Tutino Giuseppe
  \item Bressan Alessandro
\end{itemize}

\section{Contenuto della riunione}
\subsection{Ordine del giorno}
\begin{enumerate}
  \item Decidere i principali canali di comunicazione;
  \item Discutere dei capitolati presentatati;
  \item Trovare un nome e un logo;
  \item Creare una mail;
  \item Creare una repository in cui inserire la documentazione;
  \item Varie ed eventuali.
\end{enumerate}

\subsection{Riassunto}
Primo incontro interno fra i membri del team da remoto. Per prima cosa sono state discusse le opinioni di ciascuno sui capitolati, individuando le preferenze attraverso una votazione.
Successivamente sono stati scelti di comune accordo nome, logo e indirizzo di posta elettronica del team.
È inoltre stata creata l'organizzazione su GitHub che conterrà le varie repository\footnote{\url{https://github.com/orgs/techminds-unipd}}.
Ciascun componente del team si impegna a leggere attentamente i capitolati per trarre eventuali domande da porre ai proponenti. \\\\
Fissato prossimo incontro il giorno 17/10/2024 alle 14:45.

\section{Changelog}
\begin{tabularx}{1.0\textwidth} {
  | >{\centering\arraybackslash}X
  | >{\centering\arraybackslash}X
  | >{\centering\arraybackslash}X
  | >{\centering\arraybackslash}X | }
 \hline
 \textbf{Versione} & \textbf{Data} & \textbf{Descrizione} & \textbf{Autore} \\
 \hline
 1.0 & 15/10/2024 & Prima versione & Partecipanti\\
 \hline
 1.1 & 27/10/2014 & Modifica prima pagina, integrazione contenuto mancante. & Lazzarin Tommaso \\
\hline
\end{tabularx}

\end{document}
