\documentclass[10pt]{article}
\usepackage{graphicx}
\usepackage[a4paper, margin=1in]{geometry}
\usepackage{helvet}
\usepackage{hyperref}
\usepackage[italian]{babel}
\usepackage{tabularx}
\usepackage{tikz}
\hypersetup{
    colorlinks=true,
    linktoc=all,
    linkcolor=black
}

\title{Verbale esterno del meeting in data 21/10/2024\\ \large{Incontro informativo con Zucchetti}}
\author{Tech Minds \\ \href{mailto:techminds.unipd@gmail.com}{techminds.unipd@gmail.com}}
\date{}

\begin{document}

\maketitle

\begin{figure}[h]
    \centering
    \includegraphics[width=0.4\linewidth]{../../../../assets/logo.png}
\end{figure}

\tableofcontents{\newpage}

\section{Durata e partecipanti}
\subsection{Durata}
\begin{itemize}
  \item Inizio: 14:30
  \item Fine: 15:25
\end{itemize}
\subsection{Partecipanti}
\begin{itemize}
    \item \textbf{Piccoli Gregorio} (Zucchetti)
    \item Lazzarin Tommaso
    \item Salviato Leonardo
    \item Vallotto Caterina
    \item Tutino Giuseppe
    \item Bressan Alessandro
    \item Squarzoni Matteo
\end{itemize}

\section{Riassunto}
\subsection{Introduzione}
Verbale relativo all'incontro informativo con il referente del capitolato ArtificialQI proposto dall'azienda Zucchetti.
L'incontro si è tenuto sulla piattaforma Google Meet previo accordo tramite mail. \\
Oltre alle domande si è discusso anche dello stato attuale degli LLM e delle metodologie utilizzate per valutarli. Sono state mostrate anche varie risorse da consultare riguardanti l'argomento trattato (tra cui LM Studio\footnote{\url{https://lmstudio.ai/}}).
\subsection{Domande e risposte}
Di seguito sono riportate le domande poste al proponente con un riassunto delle relative risposte.
\begin{enumerate}
    \item \textbf{Come verrebbe organizzata la supervisione dell'avanzamento del progetto da parte del proponente?}\\
        Oltre agli incontri in corrispondenza delle valutazioni dei docenti, l'azienda si rende disponibile a colloqui sia in presenza che da remoto per eventuali dubbi o correzioni.
    \item \textbf{Le domande dovranno riguardare argomenti specifici?}\\
        Le domande e le relative risposte saranno fornite dall'utilizzatore del sistema. Durante la fase di sviluppo le domande potranno essere scelte liberamente dal gruppo.
    \item \textbf{Ci sono delle tecnologie che consigliate per realizzare l'interfaccia utente?}\\
        Non proponiamo l'uso di tecnologie specifiche ma suggeriamo fortemente di usare framework o librerie ampiamente diffuse (come React o Svelte) per evitare rallentamenti e problemi di incompatibilità.
    \item \textbf{Abbiamo a disposizione delle infrastrutture cloud (AWS, Azure, ecc.)?}\\
        Il progetto non prevede la messa in produzione dell'applicazione quindi non è necessario avere infrastrutture a disposizione. 
        Tuttavia l'azienda mette a disposizione delle macchine su cui far girare i modelli da testare su richiesta del gruppo.
    \item \textbf{Nel caso in cui per la valutazione si usi un LLM\footnote{Large Language Model} di capacità superiore, questo LLM deve essere prestabilito o si lascia la possibilità di scelta ad ogni esecuzione?}\\
        A differenza dei progetti che si trovano in rete, che usano un LLM considerato di capacità superiore per valutare le risposte, il progetto verte sul confronto tra la risposta ricevuta e la risposta attesa fornita dall'utente.
    \item \textbf{Due risposte diverse ma comunque giudicate corrette possono contribuire con un punteggio diverso alla valutazione del modello testato oppure il punteggio sarà sempre uguale se la risposta viene giudicata corretta?}\\
        Sarà compito del gruppo scegliere come implementare l'algoritmo di valutazione.
    \item \textbf{Nel caso di uso di un'altra LLM per la valutazione delle risposte, come si pensa di gestire le allucinazioni e le risposte che contengano informazioni aggiuntive?}\\
        Dato che la valutazione verte sul confronto tra due risposte da parte del LLM giudicante, le allucinazioni risultano meno frequenti rispetto ad una valutazione diretta della risposta. In ogni caso è possibile modificare il parametro della temperatura per cercare di ridurre questo problema.
\end{enumerate}

\section{Changelog}
\begin{tabularx}{0.8\textwidth} {
  | >{\centering\arraybackslash}X
  | >{\centering\arraybackslash}X
  | >{\centering\arraybackslash}X
  | >{\centering\arraybackslash}X | }
 \hline
 \textbf{Versione} & \textbf{Data} & \textbf{Descrizione} & \textbf{Autore} \\
 \hline
 1.0 & 21/10/2024 & Prima versione & Il gruppo Tech Minds\\
\hline
\end{tabularx}
\\ \\ 
\begin{flushright}
    \textbf{Il proponente},  % Nome della persona che firma
    Gregorio Piccoli \\
    \vspace{0.5cm}
    \begin{tikzpicture}
        \draw (0,0) -- (5,0);
    \end{tikzpicture}
\end{flushright}

\end{document}
