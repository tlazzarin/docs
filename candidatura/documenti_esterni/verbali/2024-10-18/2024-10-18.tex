\documentclass[10pt]{article}
\usepackage{graphicx}
\usepackage[a4paper, margin=1in]{geometry}
\usepackage{helvet}
\usepackage{hyperref}
\usepackage[italian]{babel}
\usepackage{tabularx}
\usepackage{tikz}
\hypersetup{
    colorlinks=true,
    linktoc=all,
    linkcolor=black
}

\title{Verbale esterno del meeting in data 18/10/2024\\ \large{Incontro informativo con Sanmarco Informatica}}
\author{Tech Minds \\ \href{mailto:techminds.unipd@gmail.com}{techminds.unipd@gmail.com}}
\date{}

\begin{document}

\maketitle

\begin{figure}[h]
    \centering
    \includegraphics[width=0.4\linewidth]{../../../../assets/logo.png}
\end{figure}

\tableofcontents{\newpage}

\section{Durata e partecipanti}
\subsection{Durata}
\begin{itemize}
  \item Inizio: 16:30
  \item Fine: 17:00
\end{itemize}
\subsection{Partecipanti}
\begin{itemize}
    \item \textbf{Beggiato Alex} (Sanmarco Informatica)
    \item Lazzarin Tommaso
    \item Salviato Leonardo
    \item Vallotto Caterina
    \item Tutino Giuseppe
    \item Bressan Alessandro
    \item Corradin Samuele
\end{itemize}

\section{Riassunto}
\subsection{Introduzione}
Verbale relativo all'incontro informativo con il referente del capitolato 3Dataviz proposto dall'azienda Sanmarco Informatica.
L'incontro si è tenuto sulla piattaforma Google Meet previo accordo tramite mail.

\subsection{Domande e risposte}
Di seguito sono riportate le domande poste al proponente con un riassunto delle relative risposte.
\begin{enumerate}
    \item \textbf{In che modo l'azienda supporterà il gruppo durante lo svolgimento del progetto?} \\
        Le modalità specifiche di supporto verranno concordate una volta aggiudicato il capitolato. 
        Sicuramente l'azienda fornirà un supporto maggiore nelle fasi di analisi dei requisiti e progettazione; durante la fase di implementazione sarà disponibile per dei chiarimenti su eventuali difficoltà. Verranno fissati degli incontri periodici per monitorare l'avanzamento del progetto.
    \item \textbf{L'azienda mette a disposizione infrastrutture cloud utili per contenere l'applicativo web ed eventuali dati?} \\
            L'azienda ritiene che non sia necessario fornire tali infrastrutture dato che esistono varie alternative gratuite sul web.
    \item \textbf{La struttura di presentazione della pagina in che modo sarà composta? Oltre al grafico 3D e all'eventuale tabella dei dati cos'altro potrebbe essere necessario visualizzare?}\\
        È necessario implementare la vista tabulare dei dati come requisito obbligatorio. Opzionalmente è possibile implementare l'interazione tra tabella e grafico in modo consistente.
    \item \textbf{Gli istogrammi possono assumere valori negativi? Se sì, si richiede la navigazione tra gli istogrammi a 360°? }\\
        Si, i valori negativi sono ammissibili ed è necessaria una libera navigazione del grafico.
    \item \textbf{Il sito deve essere pensato anche per uso mobile?} \\
        Non è una richiesta presente nei requisiti obbligatori. Tuttavia, utilizzando le tecnologie proposte adattare la webapp anche su mobile non richiede un grande impegno.
    \item \textbf{Sarà necessario inserire anche delle funzioni di personalizzazione (per esempio set di colori, o determinati dati da ‘escludere’) da parte dell’utente?} \\
        La personalizzazione dell'interfaccia non è necessaria ma opzionale.
    \item \textbf{Il sito deve essere testato per funzionare sulla maggior parte dei browser attuali oppure ha un target particolare?} \\
        Viste le tecnologie utilizzate la webapp sarà automaticamente compatibile con tutti i browser moderni.
    \item \textbf{Si deve anche tenere conto delle sessioni utente/profili registrati in modo da salvare questi istogrammi in file progetto condivisibili/scaricabili?} \\
        Sarebbe preferibile avere un welcome screen in cui è possibile scegliere il dataset da visualizzare. Non è richiesto l'uso di account.
    \item \textbf{L’utente può influenzare il database (aggiungere dati)?} \\
        La webapp si occupa solo della visualizzazione dei dati, non della loro modifica.
    \item \textbf{Ci sono dei limiti alla quantità di dati rappresentabili?} \\
        I limiti verranno decisi durante l'analisi dei requisiti. In ogni caso è fortemente consigliato scegliere un limite superiore adeguato.
    \item \textbf{Qual è una caratteristica che un gruppo dovrebbe avere per realizzare un buon prodotto?}\\
        Avere un pizzico di fantasia per rendere il prodotto più riconoscibile e incuriosire l'utente.
\end{enumerate}

\section{Changelog}
\begin{tabularx}{0.8\textwidth} {
  | >{\centering\arraybackslash}X
  | >{\centering\arraybackslash}X
  | >{\centering\arraybackslash}X
  | >{\centering\arraybackslash}X | }
 \hline
 \textbf{Versione} & \textbf{Data} & \textbf{Descrizione} & \textbf{Autore} \\
 \hline
 1.0 & 18/10/2024 & Prima versione & Il gruppo Tech Minds\\
\hline
\end{tabularx}
\\ \\ 
\begin{flushright}
    \textbf{Il proponente},  % Nome della persona che firma
    Alex Beggiato \\
    \vspace{0.5cm}
    \begin{tikzpicture}
        \draw (0,0) -- (5,0);
    \end{tikzpicture}
\end{flushright}

\end{document}
