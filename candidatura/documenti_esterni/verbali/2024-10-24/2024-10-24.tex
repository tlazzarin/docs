\documentclass[10pt]{article}
\usepackage{graphicx}
\usepackage[a4paper, margin=1in]{geometry}
\usepackage{helvet}
\usepackage{hyperref}
\usepackage[italian]{babel}
\usepackage{tabularx}
\usepackage{tikz}
\hypersetup{
    colorlinks=true,
    linktoc=all,
    linkcolor=black
}

\title{Verbale esterno del meeting in data 24/10/2024\\ \large{Incontro informativo con Var Group}}

\author{Tech Minds \\ \href{mailto:techminds.unipd@gmail.com}{techminds.unipd@gmail.com}}
\date{}

\begin{document}

\maketitle

\begin{figure}[h]
    \centering
    \includegraphics[width=0.3\linewidth]{../../../../assets/logo.png}
\end{figure}

\tableofcontents{\newpage}

\section{Durata e partecipanti}
\subsection{Durata}
\begin{itemize}
  \item Inizio: 17:00
  \item Fine: 17:35
\end{itemize}
\subsection{Partecipanti}
\begin{itemize}
    \item \textbf{Stefano Dindo} (Var Group)
    \item Lazzarin Tommaso
    \item Corradin Samuele
    \item Salviato Leonardo
    \item Vallotto Caterina
    \item Tutino Giuseppe
    \item Bressan Alessandro
    \item Squarzoni Matteo
\end{itemize}

\section{Riassunto}
\subsection{Introduzione}
Verbale relativo all'incontro informativo con il referente del capitolato C3 proposto dall'azienda Var Group.
L'incontro si è tenuto sulla piattaforma Google Meet previo accordo tramite mail. \\
\subsection{Domande e risposte}
\begin{enumerate}
    \item \textbf{Nella presentazione è citato React per le interfacce web, in che modo potrebbe essere utile un quando si dispone già di un client nativo?} \\
    È stato fatto per dare una scelta più vasta al gruppo nel caso conoscesse già le varie tecnologie web.
    \item \textbf{Cosa succede se l'utente richiede un'automazione non prevista dai blocchi definiti?} \\
    In quel caso l'applicazione dovrebbe avvisare l'utente che tale automazione non è definita.
    \item \textbf{Come fa il LLM a capire ciò che chiede l'utente?}\\
    Il Large Language Model verrà addestrato sulle API disponibili prima della sua invocazione.
    \item \textbf{Dato che gli step del workflow sono in linguaggio naturale e il software deve creare la logica di automazione, si dovrebbe far generare del codice al LLM (con delle indicazioni precise e delle funzioni limitate a disposizione) per poi eseguirlo?}\\
    Il LLM andrà a generare del codice e sarà compito del gruppo trovare un modo per verificare che il codice generato sia corretto.
    \item \textbf{Ha citato N8N, in che modo dovremmo ispirarci per il progetto?}\\
    In questo progetto la lista delle funzionalità delle varie applicazioni è nascosta all'utente e si predilige un'interazione attraverso il linguaggio naturale. In ogni caso questo progetto è stato fornito per dare l'idea di come realizzare l'interfaccia.
    \item \textbf{In che modo l'azienda darà supporto al gruppo?}\\
    Sarà organizzata una sessione di design thinking in presenza e verrà fornita una formazione sulle tecnologie utilizzate. Verranno inoltre organizzati degli incontri periodici per verificare l'avanzamento e lo stato del progetto.
    \item \textbf{È la prima volta che partecipate a questa iniziativa o siete già stati proponenti in passato?}\\
    La nostra azienda collabora già da molto tempo con l'università.
\end{enumerate}
\section{Changelog}
\begin{tabularx}{0.8\textwidth} {
  | >{\centering\arraybackslash}X
  | >{\centering\arraybackslash}X
  | >{\centering\arraybackslash}X
  | >{\centering\arraybackslash}X | }
 \hline
 \textbf{Versione} & \textbf{Data} & \textbf{Descrizione} & \textbf{Autore} \\
 \hline
 1.0 & 24/10/2024 & Prima versione & Lazzarin Tommaso\\
\hline
\end{tabularx}
\\ \\ 
\begin{flushright}
    \textbf{Il proponente},  % Nome della persona che firma
    Stefano Dindo \\
    \vspace{0.5cm}
    \begin{tikzpicture}
        \draw (0,0) -- (5,0);
    \end{tikzpicture}
\end{flushright}
\end{document}