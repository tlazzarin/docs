\documentclass[10pt]{article}
\usepackage{graphicx}
\usepackage[a4paper, margin=1in]{geometry}
\usepackage{helvet}
\usepackage{hyperref}
\usepackage[italian]{babel}
\usepackage{tabularx}
\hypersetup{
    colorlinks=true,
    linktoc=all,
    linkcolor=black
}
\setlength{\skip\footins}{2cm}
\usepackage{xcolor}
\usepackage{sectsty}

\title{\textbf{Documento di studio dei capitolati}}
\author{\href{mailto:techminds.unipd@gmail.com}{techminds.unipd@gmail.com}}
\date{}

\begin{document}

\begin{figure}
    \centering
    \includegraphics[width=0.8\linewidth]{../../../assets/logo_upscaled.png}
\end{figure}
\maketitle
\begin{center}

  \textbf{Sommario}\\
  \vspace{3mm}
  Studio dei 9 capitolati proposti per il corso di Ingegneria del Software - Anno 2024/2025.
\end{center}

\newpage
\setcounter{tocdepth}{2}
\tableofcontents{\newpage}

\section{Introduzione}
\subsection{Scopo del documento}
Questo documento vuole fornire una panoramica dei capitolati presentati dalle aziende secondo il punto di vista di ciascun componente del gruppo. Dopo le discussioni avvenute, la preferenza unanime è il capitolato 3.
\subsection{Fonte dei documenti}
I documenti di presentazione dei capitolati sono presenti nel sito del docente \footnote{\url{https://www.math.unipd.it/~tullio/IS-1/2024/Progetto/Capitolati.html}}.

\subsectionfont{\color{orange}}  % sets colour of sections

\section{Valutazioni}
\subsection{C1 - ArtificialQI (Zucchetti)}
\subsubsection{Descrizione generale}
Il proponente richiede di sviluppare un applicativo in grado di valutare le risposte generate da vari LLM, al fine di poter visualizzare i risultati e comprendere il loro comportamento dato un set di domande e risposte. Il tema alla base del capitolato fa riflettere se sia strettamente necessario un modello potente come ``ChatGPT-4-turbo" dal numero di parametri molto alto, oppure se siano sufficienti, per ottenere dei risultati accettabili, anche altri modelli con meno parametri.
\subsubsection{Funzionalità obbligatorie richieste}
\begin{itemize}
    \item Archiviazione di una lista di domande e delle risposte attese;
    \item Programma di esecuzione del test che attraverso una API ponga le domande ad un programma esterno e ne registri la risposta;
    \item Programma di valutazione della correttezza/verosimiglianza delle risposte ricevute;
    \item Procedura di presentazione dei risultati dell’esecuzione del test;
    \item I punti precedenti devono essere integrati in un unico sistema che permetta di utilizzarli come parte di un insieme e non forniti come utility separate.
\end{itemize}
\subsubsection{Funzionalità opzionali richieste}
\begin{itemize}
    \item Archiviazione dei risultati dei test;
    \item Confronto automatico tra run di test diverse;
    \item Archiviazione di set di domande/risposte;
    \item Gestione della configurazione delle chiamate alle API esterne.
\end{itemize}
\subsubsection{Tecnologie da utilizzare}
Non sono state proposte tecnologie specifiche da utilizzare all'interno del documento che descrive il capitolato, tuttavia in seguito ad un incontro da remoto con il proponente sono stati consigliati l'utilizzo di framework ampiamente diffusi come React o Svelte per evitare rallentamenti e problemi di incompatibilità.
\subsubsection{Aspetti positivi}
\begin{itemize}
    \item Possibilità di fare ricerca per valutare se la risposta ottenuta dal programma chiamato via API sia effettivamente compatibile con la risposta attesa;
    \item Approfondire il tema degli LLM, il quale è molto diffuso al giorno d'oggi;
    \item Il progetto ha una parte centrale di esplorazione delle tecnologie.
\end{itemize}
\subsubsection{Aspetti negativi e rischi}
\begin{itemize}
    \item Il punto principale del capitolato è lo sviluppo del programma di valutazione della correttezza / verosimiglianza delle risposte ricevute dei requisiti obbligatori, che a detta dello stesso proponente può essere complicato;
    \item L'azienda da disponibilità per 3 appalti per i gruppi del primo lotto e altri 3 gruppi del secondo lotto, potrebbero dunque non essere tempestivi nelle risposte dato l'alto numero di gruppi da seguire.
\end{itemize}
\subsubsection{Aiuto fornito dall'azienda}
L'azienda si rende disponibile per incontri in presenza o remoto a seconda della volontà degli studenti. Potrà essere fornito un programma esterno che espone delle API da testare. Infine il proponente metterà a disposizione delle macchine su cui far girare i modelli da testare su richiesta del gruppo.
\subsubsection{Contatti avuti con l'azienda}
Il gruppo ha fissato un colloquio informativo, sulla piattaforma Google Meet, per un approfondimento riguardo i temi del capitolato. Il proponente si è prontamente reso disponibile per rispondere a tutte le nostre domande. L'incontro è avvenuto il 21/10/2024 e il verbale è presente nell'omonima cartella\footnote{\url{https://github.com/techminds-unipd/docs/blob/main/candidatura/documenti_esterni/verbali/2024-10-21/2024-10-21-Firmato.pdf}}.

\subsubsection{Conclusione}
La proposta dell'azienda Zucchetti non ci aveva colpito dopo la presentazione tenutasi in classe, tuttavia in seguito al colloquio di approfondimento nel quale il gruppo ha compreso meglio la finalità del progetto, è stato rivalutato in maniera positiva. Nonostante questo il gruppo non ritiene questo capitolato in linea con i propri interessi.
\\\\
\subsection{C2 - Vimar GENIALE}
\subsubsection{Descrizione generale}
Il progetto richiede di creare un applicativo dotato di un’interfaccia che permetta, grazie ad un LLM, di fare richieste in linguaggio naturale riguardanti le specifiche dei vari prodotti Vimar presenti nel sito web dell’azienda, in modo tale da facilitare la ricerca di informazioni da parte dei professionisti del settore e supportarli nell’installazione dei dispositivi Vimar.

\subsubsection{Funzionalità obbligatorie richieste}
L’azienda richiede di realizzare un’infrastruttura cloud-ready che contenga un applicativo web responsive per l’utente finale e un applicativo server per la gestione dei dati, con documentazione annessa.
In particolare:

\begin{itemize} 
    \item L’applicativo web: 
    \begin{itemize}
        \item deve essere responsive e funzionare su smartphone, tablet e desktop via browser;
        \item deve avere un sistema di conversazione libera in lingua italiana;
        \item deve avere un sistema di feedback da parte dell’utente sulla risposta ottenuta;
        \item deve avere una seziona protetta dedicata agli amministratori.
    \end{itemize}
    \item L’applicativo server: 
    \begin{itemize}
        \item deve avere un sistema di estrazione e raccolta dati (database) dal sito web dell’azienda (obbligatori prodotti impianto Smart e Domotico);
        \item deve avere un sistema di indicizzazione dei dati raccolti;
        \item deve avere un componente di interrogazione che si interfacci con il sistema di indicizzazione e il modello AI (LLM open source).
    \end{itemize}
    \item L’infrastruttura cloud-ready: 
    \begin{itemize}
        \item deve avere un componente per l’applicativo web che si interfacci con l’applicativo server;
        \item deve utilizzare la tecnologia container;
        \item deve essere realizzata con il principio IaC (Infrastructure as Code).
    \end{itemize}
\end{itemize}
L’azienda inoltre richiede l’implementazione di test di unità, integrazione e end-to-end basati sui requisiti con una copertura almeno del 75\% per i primi due e almeno dell’80\% per il terzo.
\subsubsection{Funzionalità opzionali richieste}
Per l’applicativo web opzionalmente si può implementare un sistema di conversazione guidata, un sistema di suggerimenti per domande successive e la visione delle fonti usate per le risposte attraverso dei link di riferimento, mentre nell’applicativo server si può aggiungere un controllo sull’output del componente di interrogazione, oltre alla possibilità di estendere il database con i prodotti di un impianto tradizionale.
Una caratteristica opzionale per l’infrastruttura è la sua realizzazione su AWS.

\subsubsection{Tecnologie da utilizzare}
\begin{itemize}
\item Docker per l’infrastruttura cloud;
\item Git per il versionamento del repository di lavoro;
\item LLM per il componente di interrogazione.
\end{itemize}
Altri suggerimenti da parte dell’azienda:
\begin{itemize}
\item Flask, Angular, VueJS (applicativo web responsive);
\item Python;
\item Scrapy, OCRmyPDF (estrazione dati sito web);
\item PostgreSQL, TimescaleDB, InfluxDB (database);
\item Llama 3.1, Mistral, Bert, Phi (LLM open source).
\end{itemize}

\subsubsection{Aspetti positivi}
\begin{itemize}
    \item Molta disponibilità da parte dell’azienda sia in ambito formativo sia nel controllo dell’avanzamento del progetto;
\item Capitolato chiaro ed esaustivo;
\item Possibilità di interfacciarsi con un ampio insieme di tecnologie.

\end{itemize}
\subsubsection{Aspetti negativi e rischi}
Tema comune ad altri capitolati proposti.
\subsubsection{Aiuto fornito dall'azienda}
L’azienda fornisce la possibilità di organizzare molti incontri, in particolare un SAL (Stato Avanzamento Lavori) bisettimanale di un’ora che può diventare settimanale di mezz’ora e altre riunioni per approfondire le tecnologie utilizzate nel progetto o semplicemente per necessità. Vengono inoltre richiesti almeno due incontri in presenza con il gruppo per la consegna dei materiali e il collaudo della soluzione.

\subsubsection{Contatti avuti con l'azienda}
L’azienda è stata contattata via email per alcuni dubbi sulle funzionalità presentate nel capitolato, che alla fine si sono rivelati essere solamente altre funzionalità opzionali.

\subsubsection{Conclusione}
Nonostante i vari aspetti positivi il gruppo ha preferito optare per un capitolato con un tema più particolare.
\\\\
\subsection{C3 - Automatizzare le routine digitali (Var Group)}
\subsubsection{Descrizione generale}
Vargroup propone di costruire un software che, sfruttando l’intelligenza artificiale generativa, sia in grado di automatizzare alcune delle azioni che gli utenti compiono quotidianamente negli applicativi che saranno supportati. Un utente può creare un workflow selezionando i blocchi di interesse e collegandoli tramite un’interfaccia drag and drop. Ogni collegamento indica un'automazione descritta in linguaggio naturale che sarà interpretata dal modello.
\subsubsection{Funzionalità obbligatorie richieste}
\begin{itemize}
    \item Creazione di un client per disegnare i flussi di automazione su piattaforma Windows e/o Mac;
    \item Tale client deve essere dotato di un’interfaccia drag and drop e di un’interfaccia conversazionale per la definizione delle automazioni;
    \item Creazione di almeno tre blocchi di automazione;
    \item Creazione di un’infrastruttura cloud su AWS e integrazione con il client;
    \item Individuazione dei limiti e dei difetti della soluzione sviluppata;
    \item Documentazione;
    \item Bug reporting;
    \item Implementazione di test automatici con una copertura almeno del 70\%;
    \item Sviluppo dell’applicativo con un’architettura modulare per consentire un’estensione futura della documentazione.
\end{itemize}
\subsubsection{Funzionalità opzionali richieste}
Non sono richieste esplicitamente funzionalità opzionali.
\subsubsection{Tecnologie da utilizzare}
Il proponente richiede l’utilizzo delle seguenti tecnologie:
\begin{itemize}
    \item Sviluppo del client: 
    \begin{itemize}
        \item Python o C\# su Windows e Swift e Swift UI su MacOS;
        \item MongoDB o altro database locale;
        \item React per lo sviluppo di una webapp.
    \end{itemize}
    \item Sviluppo delle API cloud: 
    \begin{itemize}
        \item NodeJS;
        \item Python;
        \item Typescript.
    \end{itemize}
\end{itemize}
\subsubsection{Aspetti positivi}
\begin{itemize}
    \item Progetto interessante e che può avere un impatto positivo nelle attività lavorative e nella vita privata;
    \item Utilizzo di diverse tecnologie molto diffuse;
    \item Opportunità di formazione in azienda sulle tecnologie impiegate;
    \item Apprendimento del design thinking tramite una sessione guidata in azienda;
    \item Il progetto è la giusta combinazione fra sviluppo e innovazione;
    \item Più membri del gruppo sono interessati all’apprendimento di AWS.
\end{itemize}
\subsubsection{Aspetti negativi e rischi}
\begin{itemize}
    \item Progetto esplorativo volto a riconoscere i limiti delle tecnologie in tale ambito;
    \item L’utilizzo proficuo e consapevole di AWS richiede una buona conoscenza dello strumento.
\end{itemize}
\subsubsection{Aiuto fornito dall'azienda}
L’azienda si è resa disponibile a concordare insieme al gruppo il piano di controllo dello stato di avanzamento del progetto. Inoltre, si è resa disponibile a pianificare un percorso formativo sulle tecnologie più complesse.
Vargroup si è offerta di svolgere un’attività di design thinking con il gruppo e di fornire esempi su come creare i requisiti di business.
\subsubsection{Contatti avuti con l'azienda}
Il gruppo ha fissato un colloquio informativo, sulla piattaforma Google Meet, per un approfondimento riguardo i temi del capitolato. Il proponente si è prontamente reso disponibile per rispondere a tutte le nostre domande. L'incontro è avvenuto il 24/10/2024 e il verbale con le domande è disponibile nell'omonima cartella.\footnote{\url{https://github.com/techminds-unipd/docs/blob/main/candidatura/documenti_esterni/verbali/2024-10-24/2024-10-24-Firmato.pdf}}
\subsubsection{Conclusione}
A prima vista il progetto ha affascinato tutti i membri del gruppo ma, allo stesso tempo, ha sollevato dei dubbi circa la sua complessità. Il colloquio con Vargroup è stato fondamentale sotto questo frangente ed è riuscito a rassicurare il gruppo rispetto alla fattibilità del progetto. In particolare, il gruppo apprezza la volontà dell’azienda di proporre un progetto innovativo ed esplorativo, volto anche a individuare i limiti dell’applicativo da sviluppare.
\\\\
\subsection{C4 - NearYou (Sync Lab)}
\subsubsection{Descrizione generale}
Il proponente richiede la creazione di un sistema software che si occupi della creazione di campagne pubblicitarie mirate attraverso un LLM. Per farlo viene chiesto di implementare una piattaforma che riesca a gestire flussi continui di dati di geolocalizzazione e comportamentali degli utenti. Lo scopo è quello di far aumentare il valore della pubblicità percepito per gli inserzionisti, consentendo loro di avere una maggiore precisione nel targeting.
\subsubsection{Funzionalità obbligatorie richieste}
\begin{itemize}
    \item Simulatori di dati geospaziali;
    \item Configurazione del database con storicizzazione dei dati;
    \item Implementazione di un tool per lo stream processing di più sorgenti;
    \item Implementazione di un LLM per generare gli annunci;
    \item Sviluppo di una dashboard per visualizzare i dati;
    \item Web-app che simula un utente che segue un percorso e riceve gli annunci pubblicitari;
    \item Test che dimostrino il corretto funzionamento dei servizi;
    \item Presenza di documentazione.
\end{itemize}
\subsubsection{Tecnologie da utilizzare}
Il proponente non ha imposto vincoli stretti sulle tecnologie da usare, ma ha solamente elencato i tipi di software che è necessario utilizzare:
\begin{itemize}
    \item Framework per la simulazione dei dati;
    \item Broker per gestire lo stream come MQTT;
    \item Strumento per lo stream processing che preleva i dati per consegnarli al LLM;
    \item Strumento per processare le richieste tramite LLM;
    \item Database capace di elaborare dati geospaziali;
    \item Strumenti per la data visualization come Grafana;
    \item Web-app con tecnologia non specificata.
\end{itemize}
\subsubsection{Aspetti positivi}
\begin{itemize}
    \item Uso di tecnologie non ancora studiate come lo stream processing e l’elaborazione di dati geospaziali;
    \item Il proponente si rende disponibile a incontri, anche formativi, per guidare il gruppo.
\end{itemize}
\subsubsection{Aspetti negativi e rischi}
\begin{itemize}
    \item Il progetto vuole fare uso di molte tecnologie diverse, costringendo uno studio superficiale di tutte;
    \item Il progetto ha una complessità molto alta e richiede la consegna di sette componenti. È molto probabile che non si riesca a completarlo nel tempo a disposizione;
    L’argomento del progetto non stimola l’interesse dei componenti del gruppo ad approfondire l’argomento.
\end{itemize}
\subsubsection{Aiuto fornito dall'azienda}
L’azienda propone degli incontri di revisione dello stato di avanzamento. In particolare possono essere del tipo:
\begin{itemize}
    \item Analisi dei requisiti;
    \item Formazione;
    \item Verifica dello stato di avanzamento dei lavori;
    \item Dimostrazione di una live demo del prodotto.
\end{itemize}
\subsubsection{Contatti avuti con l'azienda}
L'azienda è stata contattata via email per richiedere chiarimenti sulle modalità di presentazione degli annunci, in particolare se fosse necessario sviluppare una web-app. Ci è stato risposto che il focus principale del progetto non risiede nella modalità di visualizzazione, sebbene questa sarà cruciale nel prodotto finale. In ogni caso, l'azienda propone di concordare l'utilizzo di applicativi che permettano una visualizzazione pronta all'uso.
\subsubsection{Conclusione}
Abbiamo apprezzato l’impegno di cui si fa carico SyncLab per formare i componenti del gruppo, tuttavia riteniamo che la proposta non sia adatta a noi dati gli aspetti negativi evidenziati sopra.
\\\\
\subsection{C5 - 3Dataviz (Sanmarco Informatica)}
\subsubsection{Descrizione generale}
Il progetto chiede di realizzare un'interfaccia web per la visualizzazione in forma
tridimensionale di dati tramite barre verticali (istogramma 3D) con i relativi dati di origine in forma tabulare.
\subsubsection{Funzionalità obbligatorie richieste}
\begin{itemize}
    \item Il grafico dovrà avere le funzioni classiche di un ambiente 3D (rotation, zoom, pan, auto-positioning);
    \item Possibilità di selezionare un elemento del grafico (barra) o una cella della griglia, nascondere / opacizzare le barre con i valori superiori o inferiori al valore della barra selezionata;
    \item Lasciare come unici elementi visibili / non opacizzati i top X / bottom Y valori;
    \item Possibilità di visualizzare il piano parallelo alla base, che rappresenta il valore medio globale e nascondere / opacizzare le barre con i valori superiori o inferiori al valore medio;
    \item Visualizzare i valori corrispettivi alla barra.
\end{itemize}
\subsubsection{Funzionalità opzionali richieste}
\begin{itemize}
    \item Poter visualizzare il piano parallelo alla base che rappresenti il valore medio di un singolo elemente dell’asse (x o y);
    \item Altre funzionalità emerse e valutate come utili, in fasi di test del MVP.
\end{itemize}
\subsubsection{Tecnologie da utilizzare}
\begin{itemize}
    \item Three.js;
    \item D3js;
    \item Angular;
    \item React.
\end{itemize}
\subsubsection{Aspetti positivi}
\begin{itemize}
    \item Tema particolare rispetto agli altri capitolati proposti;
    \item Unico capitolato che tratta il 3D.
\end{itemize}
\subsubsection{Aspetti negativi e rischi}
\begin{itemize}
    \item Poco supporto nella fase di sviluppo da parte dell’azienda;
    \item Le conoscenze da acquisire vengono considerate dal gruppo meno interessanti rispetto a quelle degli altri capitolati.
\end{itemize}
\subsubsection{Aiuto fornito dall'azienda}
L'azienda fornirà un supporto maggiore nelle fasi di analisi dei requisiti e progettazione mentre durante la fase di implementazione sarà disponibile solo in caso di eventuali difficoltà. In generale verranno fissati degli incontri periodici per monitorare l'avanzamento del progetto.
\subsubsection{Contatti avuti con l'azienda}
Il gruppo ha richiesto un colloquio informativo da remoto, su piattaforma Google Meet, dove il proponente si è reso disponibile a rispondere alle domande. Il verbale relativo all'incontro del 18/10/2024 è presente nell'omonima cartella nella repository\footnote{\url{https://github.com/techminds-unipd/docs/blob/main/candidatura/documenti_esterni/verbali/2024-10-18/2024-10-18-Firmato.pdf}}.
\subsubsection{Conclusione}
Sebbene a primo impatto il progetto abbia colto l'interesse generale dei membri, il gruppo ha optato per un capitolato differente in seguito ai colloqui avuti con le aziende. Tale scelta è stata dettata dal fatto che le tecnologie da utilizzare in altri capitolati sono risultate più interessanti e utili a scopo didattico.
\\\\
\subsection{C6 - Sistema di gestione di un magazzino distribuito (M31)}
\subsubsection{Descrizione generale}
Il progetto si propone di sviluppare un sistema di gestione distribuito per una rete di magazzini, con l'obiettivo di ottimizzare i livelli di scorte e minimizzare i tempi di risposta. L'architettura del sistema sarà basata su microservizi, garantendo operatività indipendente per ciascun magazzino e sincronizzazione in tempo reale con un sistema centrale.
\subsubsection{Funzionalità obbligatorie richieste}
\begin{itemize}
    \item Sincronizzazione in tempo reale dei dati di inventario tra i magazzini e il sistema centrale;
    \item Gestione autonoma dei magazzini tramite microservizi;
    \item Risoluzione dei conflitti negli aggiornamenti concorrenti;
    \item Riassortimento predittivo basato su machine learning;
    \item Monitoraggio centralizzato e reportistica;
    \item Gestione delle operazioni di trasferimento tra magazzini.
\end{itemize}
\subsubsection{Funzionalità opzionali richieste}
\begin{itemize}
    \item Protezione dei dati e della confidenzialità delle comunicazioni;
    \item Monitoraggio del sistema.
\end{itemize}
\subsubsection{Tecnologie da utilizzare}
\begin{itemize}
\item Node.js, Nest.js (TypeScript) per lo sviluppo di microservizi;
\item Go per componenti ad alte prestazioni;
\item NATS o Apache Kafka per la gestione della comunicazione tra i microservizi;
\item Google Cloud Platform, Kubernetes per l'orchestrazione;
\item MongoDB per dati non strutturati, PostgreSQL per dati strutturati;
\item Redis per il caching;
\item Angular per l'interfaccia utente.
\end{itemize}
\subsubsection{Aspetti positivi}
\begin{itemize}
    \item Il progetto è interessante perché propone un modo innovativo di gestire un complesso di magazzini;
    \item Machine learning utilizzato per un obiettivo diverso da un banale ChatBot;
    \item Supporto per il riassortimento predittivo tramite machine learning.
\end{itemize}
\subsubsection{Aspetti negativi e rischi}
L’uso di molte tecnologie può portare ad una conoscenza superficiale di ognuna o uno studio approfondito con conseguente probabilità di una consegna in ritardo.
\subsubsection{Aiuto fornito dall'azienda}
M31 fornirà supporto tecnico e metterà a disposizione un esperto per guidare l'analisi e lo sviluppo delle soluzioni. L'azienda supervisionerà la fase di analisi dello stato dell'arte.
\subsubsection{Conclusione}
Anche se a qualche componente del team ha suscitato interesse, il capitolato non è stato scelto a causa di altri considerati migliori, perciò si è preferito scegliere tra altre opzioni.
\\\\
\subsection{C7 - LLM: Assistente virtuale (Ergon)}
\subsubsection{Descrizione generale}
Il progetto riguarda lo sviluppo di un Assistente Virtuale basato su LLM (Large Language Models) per assistere i clienti nella ricerca di informazioni sui prodotti aziendali. L'obiettivo è automatizzare l'interazione cliente-produttore, fornendo risposte rapide e accurate su prodotti disponibili, riducendo la necessità di contattare specialisti umani per ogni richiesta.

\subsubsection{Funzionalità obbligatorie richieste}
\begin{itemize}
 \item Database relazionale per la gestione dei dati;
 \item Modello LLM con API per l’interrogazione;
 \item Interfaccia utente per la configurazione della piattaforma;
 \item Interfaccia utente per l'interazione tra IA e Utente;
 \item Elementi indipendenti ma in grado di dialogare tra loro;
 \item Comunicazione da/per il database (vedi tecnologie da utilizzare);
 \item Sviluppo di un'app mobile;
 \item Sviluppo backend Web per la configurazione della piattaforma;
 \item Feedback da parte dell’utente;
 \item Configurare da backend dei template relativi alle domande base degli utenti e fornire dei template di risposta;
 \item Documentazione.
\end{itemize}

\subsubsection{Funzionalità opzionali richieste}
Approfondire il feedback dell’utente con ulteriori domande.
\subsubsection{Tecnologie da utilizzare}
\begin{itemize}
    \item Sql Server Express, MySql o MariaDB;
    \item BLOOM;
    \item Falcon IA;
    \item Pythia;
    \item Italia by iGenius;
    \item Minerva;
    \item API REST;
    \item Connettori standard fonte dati ODBC;
    \item Middleware (JSON) per gestire comunicazione tra i componenti;
    \item .NET MAUI;
    \item Android.
\end{itemize}
\subsubsection{Aspetti positivi}
\begin{itemize}
    \item Offerta di corsi per l’introduzione a due tecnologie da utilizzare;
    \item Il progetto da una panoramica completa dello sviluppo software perchè chiede anche di gestire dati, integrare con API REST, sviluppare interfaccia utente, gestire la connessione tra tutti questi elementi.
\end{itemize}
\subsubsection{Aspetti negativi e rischi}
\begin{itemize}
    \item Dato che il modello richiede una fase di miglioramento continuo basata sui feedback degli utenti c’è il rischio che non ci siano i tempi tecnici necessari per consegnare un buon risultato;
    \item Tema comune ad altri capitolati proposti.
\end{itemize}
\subsubsection{Aiuto fornito dall'azienda}
\begin{itemize}
\item Supporto da parte del team di R\&D in varie fasi di sviluppo (non specificate);
\item Interazione da remoto o in presenza;
\item Due corsi online sui sistemi LLM e su .NET MAUI.

\end{itemize}
\subsubsection{Conclusione}
La proposta risulta meno interessante rispetto alle altre, soprattutto confrontandola con le altre proposte di assistente virtuale, portando quindi il gruppo a concentrarsi su altre opzioni.
\\\\

\subsection{C8 - Requirement Tracker - plugin VSCode (Bluewind)}
\subsubsection{Descrizione generale}
Requirement tracker è un plugin per VS Code che attraverso l’uso di intelligenza artificiale analizza il codice, i requisiti (inclusi manuali e datasheet di componenti hardware) e automatizza il tracciamento di essi nel codice sorgente per fornire suggerimenti, migliorandone quindi la qualità.
\subsubsection{Funzionalità obbligatorie richieste}
\begin{itemize}
    \item Architettura estensibile;
    \item Caricamento e scansione dei sorgenti per identificare i requisiti;
    \item il sistema dovrà essere configurabile per lavorare con i requisiti derivati dai vari documenti;
    \item Supporto ai linguaggi C/C++;
    \item Visualizzazione dei risultati con la possibilità di navigare e filtrare i risultati per nome del requisito, file o sezione del codice;
    \item Integrazione con API per inviare porzioni di codice e requisiti;
    \item Suggerimenti per la scrittura dei requisiti;
    \item Rendere i requisiti più chiari e specifici;
    \item Identificare info tecniche mancanti;
    \item Riformulare i requisiti per renderli conformi alle best practice di scrittura tecnica.
\end{itemize}
\subsubsection{Funzionalità opzionali richieste}
\begin{itemize}
    \item Supporto per più linguaggi, ad esempio Rust;
    \item Visualizzazione grafica avanzata;
    \item Valutazione della compatibilità dello strumento sviluppato con le specifiche per la scrittura dei requisiti rispetto ad una normativa in uso relativa alla sicurezza funzionale.
\end{itemize}
\subsubsection{Tecnologie da utilizzare}
\begin{itemize}
    \item VScode Extension API;
    \item API REST;
    \item Python o Node.js;
    \item Modelli AI pre-addestrati (GPT e simili);
    \item Ollama o simili.
\end{itemize}
\subsubsection{Aspetti positivi}
Progetto che si distingue dalle altre proposte dato che ha l'obiettivo di creare un'estensione di Visual Studio Code.
\subsubsection{Aspetti negativi e rischi}
\begin{itemize}
    \item Esposizione poco chiara del capitolato;
    \item Si reputa di difficile fattibilità la realizzazione del progetto.
\end{itemize}
\subsubsection{Aiuto fornito dall'azienda}
L'azienda si rende disponibile per incontri da remoto per aggiornamenti sul progresso e sessioni di consulenza; inoltre possono essere concordati incontri in presenza per discussioni più dettagliate, revisioni approfondite e supporto pratico.
\subsubsection{Conclusione}
L'argomento proposto dal capitolato non incontra gli interessi del gruppo, il quale per i motivi sopraelencati non ha ritenuto necessario contattare l'azienda per un colloquio di approfondimento.
\\\\
\subsection{C9 - BuddyBot (Azzurrodigitale)}
\subsubsection{Descrizione generale}
BuddyBot è un assistente virtuale che aggrega informazioni da GitHub, Confluence, Jira, Slack e Telegram utilizzando tecnologie di Intelligenza Artificiale per migliorare l'efficacia e la personalizzazione del supporto.
\subsubsection{Funzionalità obbligatorie richieste}
\begin{itemize}
    \item Piattaforma web per interfacciarsi al bot;
    \item Backend che fa uso delle API di Confluence, GitHub, Jira;
    \item Database per memorizzare le chat degli utenti;
    \item Test automatici con code coverage adeguata per garantire il corretto funzionamento;
    \item Documentazione tecnica;
    \item Sistema di bug reporting.
\end{itemize}
\subsubsection{Funzionalità opzionali richieste}
    Fonti di informazioni aggiuntive, ovvero canali di comunicazione aziendali come gruppi Slack e Telegram.
\subsubsection{Tecnologie da utilizzare}
\begin{itemize}
    \item OpenAI;
    \item Langchain;
    \item Angular;
    \item Node/NestJS;
    \item Spring Boot.
\end{itemize}
\subsubsection{Aspetti positivi}
\begin{itemize}
    \item Le figure messe a disposizione dall’azienda hanno ruoli e competenze diverse;
    \item La motivazione e gli obiettivi del progetto sono chiari e ben esposti. Ciò facilita il processo di analisi dei requisiti.
\end{itemize}
\subsubsection{Aspetti negativi e rischi}
\begin{itemize}
    \item Il progetto non sembra molto innovativo visto che esistono già da tempo dei plugin per ChatGPT o vari LLM che permettono di integrarsi con fonti esterne;
    \item Il tutto sembra semplicemente una webapp che fa da wrapper ad un LLM.
\end{itemize}
\subsubsection{Aiuto fornito dall'azienda}
L’azienda mette a disposizione quattro figure di riferimento per rispondere alle esigenze del gruppo.
\subsubsection{Conclusione}
La proposta non è stata ritenuta allo stesso livello della concorrenza visti gli aspetti negativi, quindi il gruppo ha preferito concentrarsi su altre proposte più allettanti.
\\\\
\section{Changelog}
\begin{tabularx}{1.0\textwidth} {
  | >{\centering\arraybackslash}X
  | >{\centering\arraybackslash}X
  | >{\centering\arraybackslash}X
  | >{\centering\arraybackslash}X | }
 \hline
 \textbf{Versione} & \textbf{Data} & \textbf{Descrizione} & \textbf{Autore} \\
 \hline
 1.0 & 25/10/2024 & Prima versione & Il gruppo Tech Minds\\
\hline
\end{tabularx}

\end{document}
