\documentclass[10pt]{article}
\usepackage{graphicx}
\usepackage[a4paper, margin=1in]{geometry}
\usepackage{helvet}
\usepackage{hyperref}
\usepackage[italian]{babel}
\usepackage{tabularx}
\usepackage{eurosym}
\hypersetup{
    colorlinks=true,
    linktoc=all,
    linkcolor=black
}

\title{\textbf{Lettera di presentazione}}
\author{\href{mailto:techminds.unipd@gmail.com}{techminds.unipd@gmail.com}}
\date{}

\begin{document}

\begin{figure}
    \centering
    \includegraphics[width=0.8\linewidth]{../../../assets/logo_upscaled.png}
\end{figure}
\maketitle
\begin{center}

  \textbf{Sommario}\\
  \vspace{3mm}
  Candidatura per Var Group - Automatizzare le routine digitali tramite l’intelligenza generativa.
\end{center}


\begin{center}

\vspace{25pt}
\textbf{Componenti}\\
\vspace{1.5mm}
\renewcommand{\arraystretch}{1.3}
\begin{tabularx}{0.5\textwidth} {
  | >{\centering\arraybackslash}X
  | >{\centering\arraybackslash}X
  | >{\centering\arraybackslash}X | }
 \hline
 \textbf{Nome} & \textbf{Cognome} & \textbf{Matricola} \\ 
 \hline
Alessandro & Bressan & 1224823\\
Samuele & Corradin & 2068235\\
Tommaso & Lazzarin & 2075529\\
Leonardo & Salviato & 2068222\\
Matteo & Squarzoni & 2068240\\
Giuseppe & Tutino & 2075515\\
Caterina & Vallotto & 2076434\\
\hline
\end{tabularx}

\end{center}

\newpage
\section{Motivazione della candidatura}
Il gruppo Tech Minds esprime il forte interesse di lavorare per il progetto \textit{Automatizzare le routine digitali tramite l’intelligenza generativa} dell'azienda Var Group.
Tutti i componenti del gruppo ritengono molto proficuo l'apprendimento delle tecnologie utilizzate all'interno di questo progetto, ovvero la tecnologia cloud di AWS, l'utilizzo di LLM attraverso l'uso di agenti e lo sviluppo di un client nativo attraverso uno dei linguaggi proposti (C\#, Python, Swift).
Un'ulteriore caratteristica che ha destato curiosità è il fatto che gli LLM non siano usati come un componente già pronto, ma vengano integrati in un ambiente più complesso (agente) che verrà utilizzato per generare dei workflow.
Inoltre il gruppo ritiene molto valide le modalità di supporto al progetto.

\section{Piano di lavoro}
Il gruppo dichiara di lavorare al progetto con intensità alta. I componenti comunicano l'impegno di portare a termine il progetto entro la data: \textbf{28 marzo 2025} con un costo stimato di \textbf{13.600 \euro}.
\\\\
La documentazione è disponibile al seguente link sulla piattaforma GitHub:\\
\url{https://github.com/techminds-unipd/docs}
\\\\
Al suo interno è possibile trovare i seguenti documenti:
\begin{itemize}
    \item Verbali interni ed esterni;
    \item Preventivo costi;
    \item Lettera di presentazione (questo documento);
    \item Documento di studio dei capitolati.
\end{itemize}

\section{Conclusione}
Grazie per la disponibilità,\\
Il gruppo Tech Minds

\end{document}
